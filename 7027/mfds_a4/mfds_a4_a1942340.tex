\documentclass{article}

\usepackage{amsmath}
\usepackage{titling}  % 引入titling宏包
\usepackage{fontspec}
\usepackage{xeCJK} % 如果你需要处理中文
\usepackage{cases}   % 支持 numcases 环境

% 设置中文字体
\setCJKmainfont{SimSun}  % 使用SimSun字体(请确保已安装)

% 设置主标题、副标题、作者和日期
\title{MATHS7027 Mathematical Foundations of Data Science}
\author{Dongju Ma}
\date{\today}

% 在主标题后添加副标题
\posttitle{\par\end{center}\begin{center}\large Trimester 2, 2024\end{center}\vskip 0.5em}

\begin{document}


\maketitle  % 生成标题和副标题


\section*{Assignment 4 - Question 1}

Consider $A = \begin{bmatrix}
    a & b\\
    c & d
\end{bmatrix}$, to make 
$A ^ 2 = \begin{bmatrix}
    4 & -2\\
    0 & 1
\end{bmatrix}$,
we could write an expression below:
$$
\begin{bmatrix}
    a & b\\
    c & d
\end{bmatrix}
\cdot
\begin{bmatrix}
    a & b\\
    c & d
\end{bmatrix}
=
\begin{bmatrix}
    4 & -2\\
    0 & 1
\end{bmatrix}
$$
$$
\begin{bmatrix}
    a^2 + bc & ab + bd\\
    ac + cd & bc + d^2
\end{bmatrix}
= 
\begin{bmatrix}
    4 & -2\\
    0 & 1
\end{bmatrix}
$$
\\
And then we could get the relationships of $a,b,c$ and $d$:
\begin{numcases}{}
a^2 + bc = 4 \\
ab + bd = -2 \\
ac + cd = 0 \\
bc + d^2 = 1
\end{numcases}
\\
from (3), we could learn that: $ac = -cd$, if $c \neq 0$, so $a = -d$.\\
So we can change (2) to $ab - ab = -2$ which is obviously incorrect.\\
Then $c = 0$, which means $a$ and $d$ could be any value based on (3).\\
But still with (1) and (4) we could obtain $a^2 = 4, d^2 = 1$,
So $a$ could be either 2 or -2 while $d$ could be either 1 or -1. 
And calculate (2) with all the cases, we could get:

$$
\begin{aligned}
2b + b &= -2, a = 2, d = 1\\
2b - b &= -2, a = 2, d = -1\\
-2b + b &= -2, a = -2, d = 1\\
-2b - b &= -2, a = -2, d = -1
\end{aligned}
$$

Consider a vector of $(a,d,b)$, which could be:

$$
(2,1,-\frac{2}{3}),\ (2,-1,-2),\ (-2,1,2) \ or \ (-2,-1,\frac{2}{3})
$$

So the matrix $A$ could be:

$$
\begin{bmatrix}
    2 & -\frac{2}{3}\\
    0 & 1
\end{bmatrix}
,\
\begin{bmatrix}
    2 & -2\\
    0 & -1
\end{bmatrix}
,\
\begin{bmatrix}
    -2 & 2\\
    0 & 1
\end{bmatrix}
or\
\begin{bmatrix}
    -2 & \frac{2}{3}\\
    0 & -1
\end{bmatrix}
$$

\section*{Assignment 4 - Question 2}

According to $W^{-1} (X + (YZ)^{T})$ is defined, we could know $W$ has the same row number to column number so it could be invertible. 
Assump that $W$ is an $n \times n$ matrix.\\
Next we could learn that $X$ has the same row number and column number to $(YZ)^{T}$, as far as we know that $Y$ has 4 columns 
and $Z$ has 5 columns, so $Z$ should have 4 rows and 5 columns to make $YZ$ is defined. Since $X$ has 3 columns and could be added to 
$YZ^{T}$, so $YZ^{T}$ should have 3 columns, which means $Y$ has 3 rows and 4 columns, and $YZ^{T}$ should be a $5 \times 3$ matrix.\\
So we can learn $X + (YZ)^{T}$ is a $5 \times 3$ matrix, and $W$ should have the same columns to the rows of $X + (YZ)^{T}$ which means 
$W$ is a $5\times 5$ matrix.\\
To make conclusions:
\begin{align*}
    W:\ 5 \times 5\\
    X:\ 5 \times 3\\
    Y:\ 3 \times 4\\
    Z:\ 4 \times 5 
\end{align*}

\section*{Assignment 4 - Question 3}

\large{\textbf{3(a)}}\\
Consider $\delta = 4$, we could obtain that:
\begin{align}
    x_1 + 4x_2 = 10\\
    x_1 - 2x_2 = 4  
\end{align}
Doulbe the formula (6) and add it to (5) then we could get:
$$
(1 + 2) x_1 = 18
$$
which lead to $x_1 = 6$\\
then put it into (5), we could obtain$x_2 = 1$.
\\
\large{\textbf{3(b)}}\\
To find the value of $\delta$, we could change the system to the following form:
\begin{align*}
    x_1 &= - \delta x_2 + 10\\
    x_1 &= 2 x_2 + \delta
\end{align*}
If the coeffcient of both $x_2$ in the two equtions are equal, that means the lines of the two equtions are parallel which 
leads to the no solutions of this system.
So the $\delta$ should be -2.
\\
\large{\textbf{3(c)}}\\
Substract $x_2 = 3$ into the equtions,
\begin{align*}
    x_1 + 3 \delta &= 10\\
    x_1 - 6 &= \delta
\end{align*}
With $x_1 = \delta + 6$, the first eqution would be:
$$
4\delta + 6 = 10
$$
And we could easily get $x_1 = 7,\ \delta = 1$

\section*{Assignment 4 - Question 4}
We could check the equation of each junction:\\
A:
$$
x_1 = 10 + 7 + x_4
$$
B:
$$
10 + x_3 + 3 = x_2
$$
C:
$$
7 = x_3 + x_5
$$
D:
$$
2 + x_4 + x_5 = 3 + 8
$$
Then we transform these equtions into $Ax = b$,\\
first:
\begin{align*}
    x_1 - x_4 &= 17\\
    x_2 - x_3 &= 13\\
    x_3 + x_5 &= 7\\
    x_4 + x_5 &= 9\\
\end{align*}
so we can obtain $A$ and $b$:
$$
A = 
\begin{bmatrix}
 1 & 0 & 0 & -1 & 0\\
 0 & 1 & -1 & 0 & 0\\
 0 & 0 & 1 & 0 & 1\\
 0 & 0 & 0 & 1 & 1   
\end{bmatrix} 
$$
$$
b = 
\begin{bmatrix}
    17\\
    13\\
    7\\
    9
\end{bmatrix}
$$
So the linear system should be like:
$$
\begin{bmatrix}
    1 & 0 & 0 & -1 & 0\\
    0 & 1 & -1 & 0 & 0\\
    0 & 0 & 1 & 0 & 1\\
    0 & 0 & 0 & 1 & 1   
\end{bmatrix} 
\cdot
\begin{bmatrix}
    x_1\\
    x_2\\
    x_3\\
    x_4\\
    x_5
\end{bmatrix}
=
\begin{bmatrix}
    17\\
    13\\
    7\\
    9
\end{bmatrix}
$$


\end{document}
